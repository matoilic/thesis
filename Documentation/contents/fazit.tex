\chapter{Fazit}

Insgesamt sind wir mit unserer Arbeit zufrieden und konnten unser Hauptziel zum grössten Teil erreichen. Wir können Türen in einem Livebild erkennen und deren Position und Rotation bestimmen. Die Arbeit war für uns eine grosse Herausforderung, da wir beide vorher nur wenig bis gar keine Erfahrung auf den unterschiedlichen Gebieten dieses Projekts hatten.
\noindent\paragraph{}
Probleme gibt es noch bei Erkennung von seitlich betrachteten Türen. Hier funktioniert die Erkennung noch nicht zuverlässig. Wir haben zu spät realisiert, dass wir für die perspektivische Erkennung zusätzliche Verfahren hätten evaluieren sollen, z.B. die Klassifizierung von Linien anhand der Tendenz zu einem Fluchtpunkt.

% TODO evtl. mit Resultat verschmelze?

\noindent\paragraph{}
Da wir mit unserer Arbeit in erster Linie eine Grundlage geschaffen haben, gibt es definitiv zusätzliches Optimierungs- und Erweiterungspotential. Die Algorithmen müssen generell im Hinblick auf die Rechenzeit optimiert werden. Viele Komponenten von OpenCV haben eine alternative Implementierung welche auf Nvidias CUDA aufbaut und Rechenoperationen auf die GPU auslagert. Hier kann wahrscheinlich einiges an Zeit eingespart werden. Dies muss aber zuerst evaluiert werden, da der Datenup- und download zur und von der GPU auch wieder Rechenzeit in Anspruch nimmt. Die CUDA-fähige Crossplattform-Kompilierung der Anwendung ist ebenfalls eine Herausforderung an sich. Die verfügbaren Versionen und Schnittstellen variieren von Plattform zu Plattform. Dies konnte zeitlich nicht mehr angegangen werden.
\noindent\paragraph{}
Ein wichtiger nächster Schritt wird sein, die Anwendung auf mobile Plattformen zu portieren. Dadurch, dass der Anwender die Möglichkeit haben soll sich um das Zielobjekt zu bewegen, ist dies für den Masseneinsatz ein Muss. Niemand wird es praktisch finden mit dem Notebook herumzulaufen und zu versuchen die Türe zu filmen.
\noindent\paragraph{}
Ein Blick in die Zukunft der Augmented Reality sieht ebenfalls vielversprechend aus. Für grosses Aufsehen sorgte diesbezüglich vor allem Google, die mit ihrem Produkt Glass versuchen Augmented Reality Massentauglich zu machen. Besonders in Alltagssituationen könnte Augmented Reality eine grosse Rolle einnehmen, beispielsweise durch Navigationshilfen in Autos, die nützliche Informationen direkt auf die Frontscheibe projizieren.
\noindent\paragraph{}
Aber auch für unseren Spezialfall der Augmented Reality kann man sich verschiedene Anwendungszwecke vorstellen. Das Einkaufen könnte interaktiver werden, indem Dinge wie Möbel als Modelle auf das eigene Smartphone geladen werden. Mittels einer Augmented Reality App können diese direkt in der aktuellen Umgebung von allen Seiten betrachtet werden. Dies könnte zudem mit aufkommenden Techniken wie NFC (Near Field Communication) erweitert werden: Man stelle sich eine Einkaufstour durch ein Möbelhaus seiner Wahl vor. Mittels NFC können 3D-Modelle von Möbeln direkt auf das Smartphone geladen werden. Später kann man die ausgesuchten Gegenstände in der eigenen Wohnung virtuell platzieren und dadurch eine bessere Kaufentscheidung machen.
